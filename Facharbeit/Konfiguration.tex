% Abschlussarbeit am Ende der Oberschule
% Facharbeit zum parallelen Programmierparadigma
% Thomas Mittermair
% Oberschulzentrum J. Ph. Fallmerayer Brixen

% define document class
\documentclass[11pt, a4paper, titlepage, twoside, german]{report}

% include required packages
\usepackage{fancyhdr}
\usepackage{graphicx}
\usepackage[ngerman, english]{babel}
\usepackage[utf8]{inputenc}
\usepackage[T1]{fontenc}
%\usepackage{times}
\usepackage{lmodern}
\usepackage{url}
\usepackage{appendix}
\usepackage[nottoc,notlot,notlof]{tocbibind}
\usepackage{listings}
\usepackage[pdftex, pdfauthor={Thomas Mittermair}, pdftitle={Paralleles Programmierparadigma}]{hyperref}	% Set title and author used in the pdf meta data.
\usepackage{microtype} % tweak spacing and sizing so everything fits nicely
\usepackage{amsmath} % Für die Darstellung von mathematischen, stückweise definierten Funktionen.
\usepackage[style=numeric, backend=bibtex]{biblatex}
\addbibresource{Literatur.bib} % Zuweisung der *.bib bei Biblatex.

% specify margins for two-sided print
\oddsidemargin 3cm
\evensidemargin 1cm
\textwidth 12cm

% define header format
\fancypagestyle{plain}{
	\fancyhf{} % Clear header/footer
	\fancyhead[LE,RO]{\leftmark}
	\fancyfoot[LE,RO]{\textit{Paralleles Programmierparadigma}}
	\fancyfoot[RE,LO]{\thepage}
	\fancyfoot[CE, CO]{}
}
\pagestyle{plain} % Set page style to plain.

\headheight 26pt
\renewcommand{\headrulewidth}{0.3pt}
\renewcommand{\footrulewidth}{0.3pt}

% define style for appendix header
\newcommand{\appendixheader}{ % specify header for appendix
    \fancyhf{}
    \fancyhead[LE, RO]{APPENDIX \rightmark}
    \fancyfoot[LO, RE]{Name 1, Name 2}
    \fancyfoot[RO, LE]{\thepage}
}

% code for creating empty pages
% no headers on empty pages before new chapter
\makeatletter
\def\cleardoublepage{\clearpage\if@twoside \ifodd\c@page\else
    \hbox{}
    \thispagestyle{empty}
    \newpage
    \if@twocolumn\hbox{}\newpage\fi\fi\fi}
\makeatother \clearpage{\pagestyle{empty}\cleardoublepage}

\setcounter{secnumdepth}{6} % setting level of numbering

\renewcommand{\arraystretch}{1.5} % Der Abstand zwischen den Zeilen einer Tabelle wird angepasst.

% Nun werden die Zeichen gesetzt, bei welchen Zeilenumbrüche bei URLs eingefügt werden können (von Latex automatisch).
\usepackage{url}
\usepackage{hyperref}
\usepackage{breakurl}
\def\UrlBreaks{\do\/\do-\do\_} % Die URL kann nur bei "/", "-" und "_" getrennt werden.

% Die folgenden Konfigurationen bzw. Paket-Einbindungen sorgen für einen meist schönen Blocksatz.
\sloppy
\usepackage{enumitem}
\usepackage{lipsum}