% Abschlussarbeit am Ende der Oberschule
% Facharbeit zum parallelen Programmierparadigma
% Thomas Mittermair
% Oberschulzentrum J. Ph. Fallmerayer Brixen

\renewcommand{\abstractname}{Abstract}

\begin{abstract}
	Das 21. Jahrhundert steht im Zeichen der Beschleunigung. Immer mehr Menschen führen ein immer schnelleres Leben, dies trifft auch auf das digitale Dasein zu. Die Benutzer verlangen immer schnellere Reaktionszeiten von Programmen und Webseiten und sind keineswegs gewillt zu warten.\\
	Das Problem hierbei liegt darin, dass Rechner bestimmten physikalischen Grenzen unterworfen sind, die eine weitere Beschleunigung ab einem gewissen Punkt unmöglich machen. An dieser Stelle kommt die Parallelisierung ins Spiel. Sie ist ein Ansatz, der zu einem Geschwindigkeitsgewinn führen kann, ohne dabei von der Weiterentwicklung der Hardware abhängig zu sein. Die Idee besteht darin, die großen zu berechnenden Probleme aufzuteilen und mehrere Komponenten \textit{parallel} daran arbeiten zu lassen. Dieser Ansatz der Programmierung wird in der Informatik als \textit{Paralleles Programmierparadigma} bezeichnet. Auch wenn der Ansatz zunächst trivial klingt, so verbergen sich dahinter trotzdem ungeahnte Probleme, die es zu lösen gilt.\\
	Aus diesem Grund ist es das Ziel dieser Arbeit, zunächst die theoretischen Grundlagen der Parallelisierung zu legen und dann verschiedene Ansätze anhand eines praktischen Beispiels zu vergleichen.
\end{abstract}