% Abschlussarbeit am Ende der Oberschule
% Facharbeit zum parallelen Programmierparadigma
% Thomas Mittermair
% Oberschulzentrum J. Ph. Fallmerayer Brixen

\chapter*{Zusammenfassung} % Durch den Stern (*) wird die Kapitel-Nummerierung bei diesem Kapitel weggelassen, allerdings muss der Eintrag im Inhaltsverzeichnis als Folge manuell erzeugt werden.
	\addcontentsline{toc}{chapter}{Zusammenfassung} % Eintrag im Inhaltsverzeichnis wird manuell erzeugt.

	Das \textit{Parallele Programmierparadigma} ist keinesfalls auf dem absteigenden Ast, sondern ein in Zukunft mit Sicherheit immer bedeutender werdender Ansatz. Die Messungen im Rahmen der Versuche haben gezeigt, dass sich durch das parallele Rechnen starke Performancesteigerungen mit wenig Mehraufwand auf Programmierungsebene realisieren lassen. Allerdings sollte man nicht nur auf den Geschwindigkeitsgewinn, den \textit{Speedup}, achten, sondern diese Verbesserung auch in ein Verhältnis mit dem damit einhergehenden Aufwand setzen. Beispielsweise muss bei einem Parallelrechner für jeden weiteren Prozess, der tatsächlich parallel ablaufen soll, eine neue CPU angekauft werden. Dies kann bei sehr vielen Prozessoren schnell kostspielig werden. Aus diesem Grund ist die Betrachtung der \textit{Effizienz} von enormer Bedeutung, da auch bestätigt werden konnte, dass der Einsatz von mehr Prozessen nur bis zu einem bestimmten Grad zu einer Leistungsverbesserung führt. Auch hat sich herausgestellt, dass die Prozess- bzw. Thread-Synchronisation ein zentraler Aspekt ist, der bei der Entwicklung von parallelen Programmen berücksichtigt werden muss, um funktionstüchtige Lösungen zu erarbeiten.\\
	Berücksichtigt man all die Konzepte und Problematiken, die sich bei der Entwicklung von parallel ablaufenden Programmen ergeben, und findet man auch eine angemessene Anzahl von Prozessen und damit auch Prozessoren, die zur Bearbeitung des Problems von Nöten sind, so kann die \textit{Parallele Programmierung} tatsächlich eine enorme Verbesserung mit sich bringen, da gewöhnliche CPUs heute nicht mehr mit hohen Kosten verbunden sind.\\
	Paralleles Rechnen ist aber mit Sicherheit nicht der einzige Ansatz, der zu einer Performanceverbesserung führen kann. In diesem Zusammenhang ist der \textit{Quantencomputer} zu nennen. Dabei werden die konventionellen \textit{Bits}, die nur zwei Zustände, nämlich 0 und 1, annehmen können, durch sogenannte \textit{Qubits} ersetzt, die aufgrund der besonderen Eigenschaften in der Welt der Quanten zur selben Zeit mehrere Zustände annehmen können. Neben eines Geschwindigkeitsgewinns erhofft man sich hierbei eine Reduktion des Energieaufwandes, welcher für derzeit aktive Supercomputer ein erhebliches Hindernis darstellt. \cite{Quantencomputer}